% Options for packages loaded elsewhere
\PassOptionsToPackage{unicode}{hyperref}
\PassOptionsToPackage{hyphens}{url}
%
\documentclass[
]{article}
\usepackage{amsmath,amssymb}
\usepackage{lmodern}
\usepackage{iftex}
\ifPDFTeX
  \usepackage[T1]{fontenc}
  \usepackage[utf8]{inputenc}
  \usepackage{textcomp} % provide euro and other symbols
\else % if luatex or xetex
  \usepackage{unicode-math}
  \defaultfontfeatures{Scale=MatchLowercase}
  \defaultfontfeatures[\rmfamily]{Ligatures=TeX,Scale=1}
\fi
% Use upquote if available, for straight quotes in verbatim environments
\IfFileExists{upquote.sty}{\usepackage{upquote}}{}
\IfFileExists{microtype.sty}{% use microtype if available
  \usepackage[]{microtype}
  \UseMicrotypeSet[protrusion]{basicmath} % disable protrusion for tt fonts
}{}
\makeatletter
\@ifundefined{KOMAClassName}{% if non-KOMA class
  \IfFileExists{parskip.sty}{%
    \usepackage{parskip}
  }{% else
    \setlength{\parindent}{0pt}
    \setlength{\parskip}{6pt plus 2pt minus 1pt}}
}{% if KOMA class
  \KOMAoptions{parskip=half}}
\makeatother
\usepackage{xcolor}
\usepackage[margin=1in]{geometry}
\usepackage{color}
\usepackage{fancyvrb}
\newcommand{\VerbBar}{|}
\newcommand{\VERB}{\Verb[commandchars=\\\{\}]}
\DefineVerbatimEnvironment{Highlighting}{Verbatim}{commandchars=\\\{\}}
% Add ',fontsize=\small' for more characters per line
\usepackage{framed}
\definecolor{shadecolor}{RGB}{248,248,248}
\newenvironment{Shaded}{\begin{snugshade}}{\end{snugshade}}
\newcommand{\AlertTok}[1]{\textcolor[rgb]{0.94,0.16,0.16}{#1}}
\newcommand{\AnnotationTok}[1]{\textcolor[rgb]{0.56,0.35,0.01}{\textbf{\textit{#1}}}}
\newcommand{\AttributeTok}[1]{\textcolor[rgb]{0.77,0.63,0.00}{#1}}
\newcommand{\BaseNTok}[1]{\textcolor[rgb]{0.00,0.00,0.81}{#1}}
\newcommand{\BuiltInTok}[1]{#1}
\newcommand{\CharTok}[1]{\textcolor[rgb]{0.31,0.60,0.02}{#1}}
\newcommand{\CommentTok}[1]{\textcolor[rgb]{0.56,0.35,0.01}{\textit{#1}}}
\newcommand{\CommentVarTok}[1]{\textcolor[rgb]{0.56,0.35,0.01}{\textbf{\textit{#1}}}}
\newcommand{\ConstantTok}[1]{\textcolor[rgb]{0.00,0.00,0.00}{#1}}
\newcommand{\ControlFlowTok}[1]{\textcolor[rgb]{0.13,0.29,0.53}{\textbf{#1}}}
\newcommand{\DataTypeTok}[1]{\textcolor[rgb]{0.13,0.29,0.53}{#1}}
\newcommand{\DecValTok}[1]{\textcolor[rgb]{0.00,0.00,0.81}{#1}}
\newcommand{\DocumentationTok}[1]{\textcolor[rgb]{0.56,0.35,0.01}{\textbf{\textit{#1}}}}
\newcommand{\ErrorTok}[1]{\textcolor[rgb]{0.64,0.00,0.00}{\textbf{#1}}}
\newcommand{\ExtensionTok}[1]{#1}
\newcommand{\FloatTok}[1]{\textcolor[rgb]{0.00,0.00,0.81}{#1}}
\newcommand{\FunctionTok}[1]{\textcolor[rgb]{0.00,0.00,0.00}{#1}}
\newcommand{\ImportTok}[1]{#1}
\newcommand{\InformationTok}[1]{\textcolor[rgb]{0.56,0.35,0.01}{\textbf{\textit{#1}}}}
\newcommand{\KeywordTok}[1]{\textcolor[rgb]{0.13,0.29,0.53}{\textbf{#1}}}
\newcommand{\NormalTok}[1]{#1}
\newcommand{\OperatorTok}[1]{\textcolor[rgb]{0.81,0.36,0.00}{\textbf{#1}}}
\newcommand{\OtherTok}[1]{\textcolor[rgb]{0.56,0.35,0.01}{#1}}
\newcommand{\PreprocessorTok}[1]{\textcolor[rgb]{0.56,0.35,0.01}{\textit{#1}}}
\newcommand{\RegionMarkerTok}[1]{#1}
\newcommand{\SpecialCharTok}[1]{\textcolor[rgb]{0.00,0.00,0.00}{#1}}
\newcommand{\SpecialStringTok}[1]{\textcolor[rgb]{0.31,0.60,0.02}{#1}}
\newcommand{\StringTok}[1]{\textcolor[rgb]{0.31,0.60,0.02}{#1}}
\newcommand{\VariableTok}[1]{\textcolor[rgb]{0.00,0.00,0.00}{#1}}
\newcommand{\VerbatimStringTok}[1]{\textcolor[rgb]{0.31,0.60,0.02}{#1}}
\newcommand{\WarningTok}[1]{\textcolor[rgb]{0.56,0.35,0.01}{\textbf{\textit{#1}}}}
\usepackage{longtable,booktabs,array}
\usepackage{calc} % for calculating minipage widths
% Correct order of tables after \paragraph or \subparagraph
\usepackage{etoolbox}
\makeatletter
\patchcmd\longtable{\par}{\if@noskipsec\mbox{}\fi\par}{}{}
\makeatother
% Allow footnotes in longtable head/foot
\IfFileExists{footnotehyper.sty}{\usepackage{footnotehyper}}{\usepackage{footnote}}
\makesavenoteenv{longtable}
\usepackage{graphicx}
\makeatletter
\def\maxwidth{\ifdim\Gin@nat@width>\linewidth\linewidth\else\Gin@nat@width\fi}
\def\maxheight{\ifdim\Gin@nat@height>\textheight\textheight\else\Gin@nat@height\fi}
\makeatother
% Scale images if necessary, so that they will not overflow the page
% margins by default, and it is still possible to overwrite the defaults
% using explicit options in \includegraphics[width, height, ...]{}
\setkeys{Gin}{width=\maxwidth,height=\maxheight,keepaspectratio}
% Set default figure placement to htbp
\makeatletter
\def\fps@figure{htbp}
\makeatother
\setlength{\emergencystretch}{3em} % prevent overfull lines
\providecommand{\tightlist}{%
  \setlength{\itemsep}{0pt}\setlength{\parskip}{0pt}}
\setcounter{secnumdepth}{-\maxdimen} % remove section numbering
\ifLuaTeX
  \usepackage{selnolig}  % disable illegal ligatures
\fi
\IfFileExists{bookmark.sty}{\usepackage{bookmark}}{\usepackage{hyperref}}
\IfFileExists{xurl.sty}{\usepackage{xurl}}{} % add URL line breaks if available
\urlstyle{same} % disable monospaced font for URLs
\hypersetup{
  pdftitle={Assignment 3. Working with Data Frame. Base R Style},
  hidelinks,
  pdfcreator={LaTeX via pandoc}}

\title{Assignment 3. Working with Data Frame. Base R Style}
\author{}
\date{\vspace{-2.5em}}

\begin{document}
\maketitle

\textbf{\emph{Note}:} \emph{This assignment practices working with Data
Frame using Base R.}

\textbf{\emph{How to do it?}}:

\begin{itemize}
\item
  Open the Rmarkdown file of this assignment
  (\href{assignment3.Rmd}{link}) in Rstudio.
\item
  Right under each question, insert a code chunk (you can use the hotkey
  \texttt{Ctrl\ +\ Alt\ +\ I} to add a code chunk) and code the solution
  for the question.
\item
  \texttt{Knit} the rmarkdown file (hotkey: \texttt{Ctrl\ +\ Alt\ +\ K})
  to export an html.
\item
  Publish the html file to your Githiub Page.
\end{itemize}

\textbf{\emph{Submission}}: Submit the link on Github of the assignment
to Canvas under Assignment 3.

\begin{center}\rule{0.5\linewidth}{0.5pt}\end{center}

\hypertarget{problems}{%
\subsection{Problems}\label{problems}}

\hfill\break

\begin{enumerate}
\def\labelenumi{\arabic{enumi}.}
\tightlist
\item
  Create the following data frame
\end{enumerate}

\begin{longtable}[]{@{}lll@{}}
\toprule()
Rank & Age & Name \\
\midrule()
\endhead
0 & 28 & Tom \\
1 & 34 & Jack \\
2 & 29 & Steve \\
3 & 42 & Ricky \\
\bottomrule()
\end{longtable}

\begin{Shaded}
\begin{Highlighting}[]
\NormalTok{R }\OtherTok{=} \FunctionTok{c}\NormalTok{(}\DecValTok{0}\NormalTok{,}\DecValTok{1}\NormalTok{,}\DecValTok{2}\NormalTok{,}\DecValTok{3}\NormalTok{)}
\NormalTok{A }\OtherTok{=} \FunctionTok{c}\NormalTok{(}\DecValTok{28}\NormalTok{,}\DecValTok{34}\NormalTok{,}\DecValTok{29}\NormalTok{,}\DecValTok{42}\NormalTok{)}
\NormalTok{N }\OtherTok{=} \FunctionTok{c}\NormalTok{(}\StringTok{"Tom"}\NormalTok{,}\StringTok{"Jack"}\NormalTok{,}\StringTok{"Steve"}\NormalTok{,}\StringTok{"Ricky"}\NormalTok{)}
\NormalTok{df }\OtherTok{=} \FunctionTok{cbind}\NormalTok{(R,A,N)}
\NormalTok{df}
\end{Highlighting}
\end{Shaded}

\begin{verbatim}
##      R   A    N      
## [1,] "0" "28" "Tom"  
## [2,] "1" "34" "Jack" 
## [3,] "2" "29" "Steve"
## [4,] "3" "42" "Ricky"
\end{verbatim}

\begin{enumerate}
\def\labelenumi{\arabic{enumi}.}
\setcounter{enumi}{1}
\tightlist
\item
  Use \texttt{read.csv} to import the Covid19 Vaccination data from WHO:
  \href{https://raw.githubusercontent.com/nytimes/covid-19-data/master/us-states.csv}{link}.
\end{enumerate}

\begin{Shaded}
\begin{Highlighting}[]
\NormalTok{df}\OtherTok{=}\FunctionTok{read.csv}\NormalTok{(}\StringTok{"https://raw.githubusercontent.com/nytimes/covid{-}19{-}data/master/us{-}states.csv"}\NormalTok{)}
\end{Highlighting}
\end{Shaded}

\begin{enumerate}
\def\labelenumi{\arabic{enumi}.}
\setcounter{enumi}{1}
\tightlist
\item
  Show the names of the variables in the data
\end{enumerate}

\begin{Shaded}
\begin{Highlighting}[]
\FunctionTok{names}\NormalTok{(df)}
\end{Highlighting}
\end{Shaded}

\begin{verbatim}
## [1] "date"   "state"  "fips"   "cases"  "deaths"
\end{verbatim}

\begin{enumerate}
\def\labelenumi{\arabic{enumi}.}
\setcounter{enumi}{2}
\tightlist
\item
  How many columns and rows the data have?
\end{enumerate}

\begin{Shaded}
\begin{Highlighting}[]
\FunctionTok{ncol}\NormalTok{(df)}
\end{Highlighting}
\end{Shaded}

\begin{verbatim}
## [1] 5
\end{verbatim}

\begin{Shaded}
\begin{Highlighting}[]
\FunctionTok{nrow}\NormalTok{(df)}
\end{Highlighting}
\end{Shaded}

\begin{verbatim}
## [1] 52814
\end{verbatim}

\begin{enumerate}
\def\labelenumi{\arabic{enumi}.}
\setcounter{enumi}{3}
\tightlist
\item
  How many missing values are there? Show the missing values by columns.
  What variable has the most number of missing values?
\end{enumerate}

\begin{Shaded}
\begin{Highlighting}[]
\FunctionTok{sum}\NormalTok{(}\FunctionTok{is.na}\NormalTok{(df))}
\end{Highlighting}
\end{Shaded}

\begin{verbatim}
## [1] 0
\end{verbatim}

\begin{Shaded}
\begin{Highlighting}[]
\FunctionTok{colSums}\NormalTok{(}\FunctionTok{is.na}\NormalTok{(df))}
\end{Highlighting}
\end{Shaded}

\begin{verbatim}
##   date  state   fips  cases deaths 
##      0      0      0      0      0
\end{verbatim}

\begin{Shaded}
\begin{Highlighting}[]
\FunctionTok{colMeans}\NormalTok{(}\FunctionTok{is.na}\NormalTok{(df))}
\end{Highlighting}
\end{Shaded}

\begin{verbatim}
##   date  state   fips  cases deaths 
##      0      0      0      0      0
\end{verbatim}

\begin{enumerate}
\def\labelenumi{\arabic{enumi}.}
\setcounter{enumi}{4}
\tightlist
\item
  What is the class of the \texttt{date} column. Change the
  \texttt{date} columns to \texttt{date} type using the \texttt{as.Date}
  function. Show the new class of the \texttt{date} column.
\end{enumerate}

\begin{Shaded}
\begin{Highlighting}[]
\FunctionTok{names}\NormalTok{(df)[}\DecValTok{1}\NormalTok{] }\OtherTok{\textless{}{-}} \StringTok{\textquotesingle{}Date\_report\textquotesingle{}}
\FunctionTok{class}\NormalTok{(df}\SpecialCharTok{$}\NormalTok{Date\_report)}
\end{Highlighting}
\end{Shaded}

\begin{verbatim}
## [1] "character"
\end{verbatim}

\begin{Shaded}
\begin{Highlighting}[]
\FunctionTok{str}\NormalTok{(df)}
\end{Highlighting}
\end{Shaded}

\begin{verbatim}
## 'data.frame':    52814 obs. of  5 variables:
##  $ Date_report: chr  "2020-01-21" "2020-01-22" "2020-01-23" "2020-01-24" ...
##  $ state      : chr  "Washington" "Washington" "Washington" "Illinois" ...
##  $ fips       : int  53 53 53 17 53 6 17 53 4 6 ...
##  $ cases      : int  1 1 1 1 1 1 1 1 1 2 ...
##  $ deaths     : int  0 0 0 0 0 0 0 0 0 0 ...
\end{verbatim}

\begin{enumerate}
\def\labelenumi{\arabic{enumi}.}
\setcounter{enumi}{5}
\tightlist
\item
  Capitalize the names of all the variables
\end{enumerate}

\begin{Shaded}
\begin{Highlighting}[]
\FunctionTok{names}\NormalTok{(df)[}\DecValTok{1}\NormalTok{] }\OtherTok{\textless{}{-}} \StringTok{\textquotesingle{}DATE\_REPORT\textquotesingle{}}
\FunctionTok{names}\NormalTok{(df)[}\DecValTok{2}\NormalTok{] }\OtherTok{\textless{}{-}} \StringTok{\textquotesingle{}STATE\textquotesingle{}}
\FunctionTok{names}\NormalTok{(df)[}\DecValTok{3}\NormalTok{] }\OtherTok{\textless{}{-}} \StringTok{\textquotesingle{}FIPS\textquotesingle{}}
\FunctionTok{names}\NormalTok{(df)[}\DecValTok{4}\NormalTok{] }\OtherTok{\textless{}{-}} \StringTok{\textquotesingle{}CASES\textquotesingle{}}
\FunctionTok{names}\NormalTok{(df)[}\DecValTok{5}\NormalTok{] }\OtherTok{\textless{}{-}} \StringTok{\textquotesingle{}DEATH\textquotesingle{}}
\end{Highlighting}
\end{Shaded}

\begin{enumerate}
\def\labelenumi{\arabic{enumi}.}
\setcounter{enumi}{6}
\tightlist
\item
  Find the average number of cases per day. Find the maximum cases a
  day.
\end{enumerate}

\begin{Shaded}
\begin{Highlighting}[]
\NormalTok{x}\OtherTok{=}\FunctionTok{aggregate}\NormalTok{(df}\SpecialCharTok{$}\NormalTok{CASES, }\AttributeTok{by=}\FunctionTok{list}\NormalTok{(}\AttributeTok{DATE\_REPORT=}\NormalTok{df}\SpecialCharTok{$}\NormalTok{DATE\_REPORT), }\AttributeTok{FUN=}\NormalTok{mean)}
\NormalTok{x}\SpecialCharTok{$}\NormalTok{DATE\_REPORT[x}\SpecialCharTok{$}\NormalTok{x}\SpecialCharTok{==}\FunctionTok{max}\NormalTok{(x}\SpecialCharTok{$}\NormalTok{x)]}
\end{Highlighting}
\end{Shaded}

\begin{verbatim}
## [1] "2022-10-11"
\end{verbatim}

\begin{enumerate}
\def\labelenumi{\arabic{enumi}.}
\setcounter{enumi}{7}
\tightlist
\item
  How many states are there in the data?
\end{enumerate}

\begin{Shaded}
\begin{Highlighting}[]
\FunctionTok{length}\NormalTok{(}\FunctionTok{unique}\NormalTok{(df}\SpecialCharTok{$}\NormalTok{STATE))}
\end{Highlighting}
\end{Shaded}

\begin{verbatim}
## [1] 56
\end{verbatim}

\begin{enumerate}
\def\labelenumi{\arabic{enumi}.}
\setcounter{enumi}{8}
\tightlist
\item
  Create a new variable \texttt{weekdays} to store the weekday for each
  rows.
\end{enumerate}

\begin{Shaded}
\begin{Highlighting}[]
\NormalTok{df}\SpecialCharTok{$}\NormalTok{DATE\_REPORT }\OtherTok{=} \FunctionTok{as.Date}\NormalTok{(df}\SpecialCharTok{$}\NormalTok{DATE\_REPORT)}
\NormalTok{df}\SpecialCharTok{$}\NormalTok{WEEK\_DAY}\OtherTok{=}\FunctionTok{weekdays}\NormalTok{(df}\SpecialCharTok{$}\NormalTok{DATE\_REPORT)}
\end{Highlighting}
\end{Shaded}

\begin{enumerate}
\def\labelenumi{\arabic{enumi}.}
\setcounter{enumi}{9}
\tightlist
\item
  Create the categorical variable \texttt{death2} variable taking the
  values as follows
\end{enumerate}

\begin{Shaded}
\begin{Highlighting}[]
\NormalTok{DEATH2}\OtherTok{=}\FunctionTok{c}\NormalTok{()}

\ControlFlowTok{for}\NormalTok{ (x }\ControlFlowTok{in}\NormalTok{ df}\SpecialCharTok{$}\NormalTok{DEATH)\{}
  \ControlFlowTok{if}\NormalTok{ (x}\SpecialCharTok{==}\DecValTok{0}\NormalTok{)\{}
\NormalTok{    DEATH2}\OtherTok{=}\FunctionTok{append}\NormalTok{(DEATH2,}\StringTok{"no\_death"}\NormalTok{)}
\NormalTok{  \}}
  \ControlFlowTok{else}\NormalTok{\{}
\NormalTok{    DEATH2}\OtherTok{=}\FunctionTok{append}\NormalTok{(DEATH2,}\StringTok{"death"}\NormalTok{)}
\NormalTok{  \}}
\NormalTok{\}}
\NormalTok{df}\SpecialCharTok{$}\NormalTok{DEATH2}\OtherTok{=}\NormalTok{DEATH2}
\end{Highlighting}
\end{Shaded}

\begin{itemize}
\tightlist
\item
  \texttt{has\_death} if there is a death that day
\item
  \texttt{no\_death} if there is no death that day
\end{itemize}

Find the frequency and relative frequency of \texttt{no\_death} and
\texttt{has\_death}.

\begin{enumerate}
\def\labelenumi{\arabic{enumi}.}
\setcounter{enumi}{10}
\tightlist
\item
  Find the first quartile (Q1), second quartile (Q2) and and third
  quartile (Q3) of the variable \texttt{death}. (Hint: Use the
  \texttt{summary} function)
\end{enumerate}

\begin{Shaded}
\begin{Highlighting}[]
\FunctionTok{summary}\NormalTok{(df}\SpecialCharTok{$}\NormalTok{DEATH)}
\end{Highlighting}
\end{Shaded}

\begin{verbatim}
##    Min. 1st Qu.  Median    Mean 3rd Qu.    Max. 
##       0     890    4240   10419   13014   96280
\end{verbatim}

\begin{enumerate}
\def\labelenumi{\arabic{enumi}.}
\setcounter{enumi}{11}
\tightlist
\item
  Create the categorical variable \texttt{death3} variable taking the
  values as follows
\end{enumerate}

\begin{itemize}
\item
  \texttt{low\_death} if the number of deaths smaller than the 25
  percentile (Q1)
\item
  \texttt{mid\_death} if the number of deaths from Q1 to Q3
\item
  \texttt{high\_death} if the number of deaths greater than Q3
\end{itemize}

\begin{Shaded}
\begin{Highlighting}[]
\NormalTok{DEATH3}\OtherTok{=}\FunctionTok{c}\NormalTok{()}
\NormalTok{a}\OtherTok{=}\FunctionTok{quantile}\NormalTok{(df}\SpecialCharTok{$}\NormalTok{DEATH)[}\DecValTok{2}\NormalTok{]}
\NormalTok{b}\OtherTok{=}\FunctionTok{quantile}\NormalTok{(df}\SpecialCharTok{$}\NormalTok{DEATH)[}\DecValTok{4}\NormalTok{]}

\ControlFlowTok{for}\NormalTok{ (x }\ControlFlowTok{in}\NormalTok{ df}\SpecialCharTok{$}\NormalTok{DEATH)\{}
  \ControlFlowTok{if}\NormalTok{ (x}\SpecialCharTok{\textless{}}\NormalTok{a)\{}
\NormalTok{    y}\OtherTok{=}\StringTok{"low\_death"}
\NormalTok{  \}}
  \ControlFlowTok{else} \ControlFlowTok{if}\NormalTok{ (x}\SpecialCharTok{\textless{}}\NormalTok{b)\{}
\NormalTok{    y}\OtherTok{=}\StringTok{"mid\_death"}
\NormalTok{  \}}
  \ControlFlowTok{else}\NormalTok{ \{}
\NormalTok{    y}\OtherTok{=}\StringTok{"high\_death"}
\NormalTok{  \}}
\NormalTok{  DEATH3}\OtherTok{=}\FunctionTok{append}\NormalTok{(DEATH3,y)}
\NormalTok{\}}
\NormalTok{df}\SpecialCharTok{$}\NormalTok{DEATH3}\OtherTok{=}\NormalTok{DEATH3}
\end{Highlighting}
\end{Shaded}

\begin{enumerate}
\def\labelenumi{\arabic{enumi}.}
\setcounter{enumi}{12}
\tightlist
\item
  Find the average cases in Rhode Island in 2021
\end{enumerate}

\begin{Shaded}
\begin{Highlighting}[]
\NormalTok{df}\SpecialCharTok{$}\NormalTok{YEAR }\OtherTok{\textless{}{-}} \FunctionTok{as.numeric}\NormalTok{(}\FunctionTok{format}\NormalTok{(df}\SpecialCharTok{$}\NormalTok{DATE\_REPORT,}\StringTok{\textquotesingle{}\%Y\textquotesingle{}}\NormalTok{))}
\FunctionTok{mean}\NormalTok{(df}\SpecialCharTok{$}\NormalTok{CASES[df}\SpecialCharTok{$}\NormalTok{STATE}\SpecialCharTok{==}\StringTok{"Rhode Island"}\SpecialCharTok{\&}\NormalTok{df}\SpecialCharTok{$}\NormalTok{YEAR}\SpecialCharTok{==}\StringTok{"2021"}\NormalTok{])}
\end{Highlighting}
\end{Shaded}

\begin{verbatim}
## [1] 154438.5
\end{verbatim}

\begin{enumerate}
\def\labelenumi{\arabic{enumi}.}
\setcounter{enumi}{13}
\tightlist
\item
  Find the median cases by weekdays in Rhode Island in 2021
\end{enumerate}

\begin{Shaded}
\begin{Highlighting}[]
\NormalTok{x}\OtherTok{=}\FunctionTok{subset}\NormalTok{(df,df}\SpecialCharTok{$}\NormalTok{STATE}\SpecialCharTok{==}\StringTok{"Rhode Island"}\SpecialCharTok{\&}\NormalTok{df}\SpecialCharTok{$}\NormalTok{YEAR}\SpecialCharTok{==}\DecValTok{2021}\NormalTok{)}
\NormalTok{y}\OtherTok{=}\FunctionTok{aggregate}\NormalTok{(x}\SpecialCharTok{$}\NormalTok{CASES, }\AttributeTok{by=}\FunctionTok{list}\NormalTok{(}\AttributeTok{WEEK\_DAY=}\NormalTok{x}\SpecialCharTok{$}\NormalTok{WEEK\_DAY), }\AttributeTok{FUN=}\NormalTok{median)}
\NormalTok{y}
\end{Highlighting}
\end{Shaded}

\begin{verbatim}
##    WEEK_DAY        x
## 1    Friday 152643.0
## 2    Monday 152605.0
## 3  Saturday 152578.5
## 4    Sunday 152578.5
## 5  Thursday 152683.0
## 6   Tuesday 152643.5
## 7 Wednesday 152675.5
\end{verbatim}

\begin{enumerate}
\def\labelenumi{\arabic{enumi}.}
\setcounter{enumi}{14}
\tightlist
\item
  Compare the median cases in Rhode Island in June, July, August and
  September in 2021.
\end{enumerate}

\begin{Shaded}
\begin{Highlighting}[]
\NormalTok{df}\SpecialCharTok{$}\NormalTok{MONTH }\OtherTok{\textless{}{-}} \FunctionTok{as.numeric}\NormalTok{(}\FunctionTok{format}\NormalTok{(df}\SpecialCharTok{$}\NormalTok{DATE\_REPORT,}\StringTok{\textquotesingle{}\%m\textquotesingle{}}\NormalTok{))}
\NormalTok{x}\OtherTok{=}\FunctionTok{subset}\NormalTok{(df,df}\SpecialCharTok{$}\NormalTok{STATE}\SpecialCharTok{==}\StringTok{"Rhode Island"}\SpecialCharTok{\&}\NormalTok{df}\SpecialCharTok{$}\NormalTok{YEAR}\SpecialCharTok{==}\DecValTok{2021}\NormalTok{)}
\NormalTok{y}\OtherTok{=}\FunctionTok{aggregate}\NormalTok{(x}\SpecialCharTok{$}\NormalTok{CASES, }\AttributeTok{by=}\FunctionTok{list}\NormalTok{(}\AttributeTok{MONTH=}\NormalTok{x}\SpecialCharTok{$}\NormalTok{MONTH), }\AttributeTok{FUN=}\NormalTok{median)}
\NormalTok{y[}\DecValTok{6}\SpecialCharTok{:}\DecValTok{9}\NormalTok{,]}
\end{Highlighting}
\end{Shaded}

\begin{verbatim}
##   MONTH        x
## 6     6 152319.5
## 7     7 152971.0
## 8     8 158283.0
## 9     9 167814.0
\end{verbatim}

\begin{enumerate}
\def\labelenumi{\arabic{enumi}.}
\setcounter{enumi}{15}
\tightlist
\item
  Find your own dataset, import it and implement the following functions
  on the data
\end{enumerate}

\begin{itemize}
\tightlist
\item
  head
\item
  str
\item
  names
\item
  mean, min, max
\item
  table
\item
  is.na
\item
  colSums
\item
  class
\item
  cor
\item
  by
\item
  ifelse
\item
  case\_when
\end{itemize}

\begin{Shaded}
\begin{Highlighting}[]
\NormalTok{df}\OtherTok{=}\FunctionTok{read.csv}\NormalTok{(}\StringTok{"C:}\SpecialCharTok{\textbackslash{}\textbackslash{}}\StringTok{Users}\SpecialCharTok{\textbackslash{}\textbackslash{}}\StringTok{Farida Yesmin}\SpecialCharTok{\textbackslash{}\textbackslash{}}\StringTok{Desktop}\SpecialCharTok{\textbackslash{}\textbackslash{}}\StringTok{M421 RStats}\SpecialCharTok{\textbackslash{}\textbackslash{}}\StringTok{aroshenoor.github.io}\SpecialCharTok{\textbackslash{}\textbackslash{}}\StringTok{titanic.csv"}\NormalTok{)}
\FunctionTok{head}\NormalTok{(df)}
\end{Highlighting}
\end{Shaded}

\begin{verbatim}
##   PassengerId Survived Pclass
## 1           1        0      3
## 2           2        1      1
## 3           3        1      3
## 4           4        1      1
## 5           5        0      3
## 6           6        0      3
##                                                  Name    Sex Age SibSp Parch
## 1                             Braund, Mr. Owen Harris   male  22     1     0
## 2 Cumings, Mrs. John Bradley (Florence Briggs Thayer) female  38     1     0
## 3                              Heikkinen, Miss. Laina female  26     0     0
## 4        Futrelle, Mrs. Jacques Heath (Lily May Peel) female  35     1     0
## 5                            Allen, Mr. William Henry   male  35     0     0
## 6                                    Moran, Mr. James   male  NA     0     0
##             Ticket    Fare Cabin Embarked
## 1        A/5 21171  7.2500              S
## 2         PC 17599 71.2833   C85        C
## 3 STON/O2. 3101282  7.9250              S
## 4           113803 53.1000  C123        S
## 5           373450  8.0500              S
## 6           330877  8.4583              Q
\end{verbatim}

\begin{enumerate}
\def\labelenumi{\arabic{enumi}.}
\setcounter{enumi}{16}
\tightlist
\item
  In the dataset in \#16, practice the follows. You can reuse the code
  of 16.
\end{enumerate}

\begin{itemize}
\tightlist
\item
  Create a categorical variable from a continuous variable
\item
  From one categorical variable, create a new categorical variable with
  fewer categories
\end{itemize}

If you do not have a data, you can use
\href{https://www.kaggle.com/competitions/titanic/overview}{the titanic
dataset}, which can be downloaded at \href{../data/titanic.csv}{this
link}

\end{document}
